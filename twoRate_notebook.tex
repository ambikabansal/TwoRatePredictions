\documentclass[]{article}
\usepackage{lmodern}
\usepackage{amssymb,amsmath}
\usepackage{ifxetex,ifluatex}
\usepackage{fixltx2e} % provides \textsubscript
\ifnum 0\ifxetex 1\fi\ifluatex 1\fi=0 % if pdftex
  \usepackage[T1]{fontenc}
  \usepackage[utf8]{inputenc}
\else % if luatex or xelatex
  \ifxetex
    \usepackage{mathspec}
  \else
    \usepackage{fontspec}
  \fi
  \defaultfontfeatures{Ligatures=TeX,Scale=MatchLowercase}
\fi
% use upquote if available, for straight quotes in verbatim environments
\IfFileExists{upquote.sty}{\usepackage{upquote}}{}
% use microtype if available
\IfFileExists{microtype.sty}{%
\usepackage{microtype}
\UseMicrotypeSet[protrusion]{basicmath} % disable protrusion for tt fonts
}{}
\usepackage[margin=1in]{geometry}
\usepackage{hyperref}
\hypersetup{unicode=true,
            pdftitle={Abrupt versus Gradual Motor Learning},
            pdfborder={0 0 0},
            breaklinks=true}
\urlstyle{same}  % don't use monospace font for urls
\usepackage{color}
\usepackage{fancyvrb}
\newcommand{\VerbBar}{|}
\newcommand{\VERB}{\Verb[commandchars=\\\{\}]}
\DefineVerbatimEnvironment{Highlighting}{Verbatim}{commandchars=\\\{\}}
% Add ',fontsize=\small' for more characters per line
\usepackage{framed}
\definecolor{shadecolor}{RGB}{248,248,248}
\newenvironment{Shaded}{\begin{snugshade}}{\end{snugshade}}
\newcommand{\KeywordTok}[1]{\textcolor[rgb]{0.13,0.29,0.53}{\textbf{#1}}}
\newcommand{\DataTypeTok}[1]{\textcolor[rgb]{0.13,0.29,0.53}{#1}}
\newcommand{\DecValTok}[1]{\textcolor[rgb]{0.00,0.00,0.81}{#1}}
\newcommand{\BaseNTok}[1]{\textcolor[rgb]{0.00,0.00,0.81}{#1}}
\newcommand{\FloatTok}[1]{\textcolor[rgb]{0.00,0.00,0.81}{#1}}
\newcommand{\ConstantTok}[1]{\textcolor[rgb]{0.00,0.00,0.00}{#1}}
\newcommand{\CharTok}[1]{\textcolor[rgb]{0.31,0.60,0.02}{#1}}
\newcommand{\SpecialCharTok}[1]{\textcolor[rgb]{0.00,0.00,0.00}{#1}}
\newcommand{\StringTok}[1]{\textcolor[rgb]{0.31,0.60,0.02}{#1}}
\newcommand{\VerbatimStringTok}[1]{\textcolor[rgb]{0.31,0.60,0.02}{#1}}
\newcommand{\SpecialStringTok}[1]{\textcolor[rgb]{0.31,0.60,0.02}{#1}}
\newcommand{\ImportTok}[1]{#1}
\newcommand{\CommentTok}[1]{\textcolor[rgb]{0.56,0.35,0.01}{\textit{#1}}}
\newcommand{\DocumentationTok}[1]{\textcolor[rgb]{0.56,0.35,0.01}{\textbf{\textit{#1}}}}
\newcommand{\AnnotationTok}[1]{\textcolor[rgb]{0.56,0.35,0.01}{\textbf{\textit{#1}}}}
\newcommand{\CommentVarTok}[1]{\textcolor[rgb]{0.56,0.35,0.01}{\textbf{\textit{#1}}}}
\newcommand{\OtherTok}[1]{\textcolor[rgb]{0.56,0.35,0.01}{#1}}
\newcommand{\FunctionTok}[1]{\textcolor[rgb]{0.00,0.00,0.00}{#1}}
\newcommand{\VariableTok}[1]{\textcolor[rgb]{0.00,0.00,0.00}{#1}}
\newcommand{\ControlFlowTok}[1]{\textcolor[rgb]{0.13,0.29,0.53}{\textbf{#1}}}
\newcommand{\OperatorTok}[1]{\textcolor[rgb]{0.81,0.36,0.00}{\textbf{#1}}}
\newcommand{\BuiltInTok}[1]{#1}
\newcommand{\ExtensionTok}[1]{#1}
\newcommand{\PreprocessorTok}[1]{\textcolor[rgb]{0.56,0.35,0.01}{\textit{#1}}}
\newcommand{\AttributeTok}[1]{\textcolor[rgb]{0.77,0.63,0.00}{#1}}
\newcommand{\RegionMarkerTok}[1]{#1}
\newcommand{\InformationTok}[1]{\textcolor[rgb]{0.56,0.35,0.01}{\textbf{\textit{#1}}}}
\newcommand{\WarningTok}[1]{\textcolor[rgb]{0.56,0.35,0.01}{\textbf{\textit{#1}}}}
\newcommand{\AlertTok}[1]{\textcolor[rgb]{0.94,0.16,0.16}{#1}}
\newcommand{\ErrorTok}[1]{\textcolor[rgb]{0.64,0.00,0.00}{\textbf{#1}}}
\newcommand{\NormalTok}[1]{#1}
\usepackage{graphicx,grffile}
\makeatletter
\def\maxwidth{\ifdim\Gin@nat@width>\linewidth\linewidth\else\Gin@nat@width\fi}
\def\maxheight{\ifdim\Gin@nat@height>\textheight\textheight\else\Gin@nat@height\fi}
\makeatother
% Scale images if necessary, so that they will not overflow the page
% margins by default, and it is still possible to overwrite the defaults
% using explicit options in \includegraphics[width, height, ...]{}
\setkeys{Gin}{width=\maxwidth,height=\maxheight,keepaspectratio}
\IfFileExists{parskip.sty}{%
\usepackage{parskip}
}{% else
\setlength{\parindent}{0pt}
\setlength{\parskip}{6pt plus 2pt minus 1pt}
}
\setlength{\emergencystretch}{3em}  % prevent overfull lines
\providecommand{\tightlist}{%
  \setlength{\itemsep}{0pt}\setlength{\parskip}{0pt}}
\setcounter{secnumdepth}{0}
% Redefines (sub)paragraphs to behave more like sections
\ifx\paragraph\undefined\else
\let\oldparagraph\paragraph
\renewcommand{\paragraph}[1]{\oldparagraph{#1}\mbox{}}
\fi
\ifx\subparagraph\undefined\else
\let\oldsubparagraph\subparagraph
\renewcommand{\subparagraph}[1]{\oldsubparagraph{#1}\mbox{}}
\fi

%%% Use protect on footnotes to avoid problems with footnotes in titles
\let\rmarkdownfootnote\footnote%
\def\footnote{\protect\rmarkdownfootnote}

%%% Change title format to be more compact
\usepackage{titling}

% Create subtitle command for use in maketitle
\newcommand{\subtitle}[1]{
  \posttitle{
    \begin{center}\large#1\end{center}
    }
}

\setlength{\droptitle}{-2em}

  \title{Abrupt versus Gradual Motor Learning}
    \pretitle{\vspace{\droptitle}\centering\huge}
  \posttitle{\par}
    \author{}
    \preauthor{}\postauthor{}
    \date{}
    \predate{}\postdate{}
  

\begin{document}
\maketitle

\begin{Shaded}
\begin{Highlighting}[]
\KeywordTok{plot}\NormalTok{(cars)}
\end{Highlighting}
\end{Shaded}

\includegraphics{twoRate_notebook_files/figure-latex/unnamed-chunk-1-1.pdf}

Add a new chunk by clicking the \emph{Insert Chunk} button on the
toolbar or by pressing \emph{Cmd+Option+I}.

When you save the notebook, an HTML file containing the code and output
will be saved alongside it (click the \emph{Preview} button or press
\emph{Cmd+Shift+K} to preview the HTML file).

The preview shows you a rendered HTML copy of the contents of the
editor. Consequently, unlike \emph{Knit}, \emph{Preview} does not run
any R code chunks. Instead, the output of the chunk when it was last run
in the editor is displayed.

\section{Load Packages}\label{load-packages}

\begin{Shaded}
\begin{Highlighting}[]
\KeywordTok{library}\NormalTok{(svglite)}
\KeywordTok{library}\NormalTok{(optimx)}
\KeywordTok{library}\NormalTok{(ez)}
\end{Highlighting}
\end{Shaded}

\section{Load R scripts}\label{load-r-scripts}

\begin{Shaded}
\begin{Highlighting}[]
\KeywordTok{source}\NormalTok{(}\StringTok{'orderEffects.R'}\NormalTok{) }\CommentTok{# this file deals with the order effects and runs the ANOVA}
\KeywordTok{source}\NormalTok{(}\StringTok{'ExtentofLearning.R'}\NormalTok{) }\CommentTok{# this file deals with the extent of learning and runs the ANOVA}
\KeywordTok{source}\NormalTok{(}\StringTok{'Analysis.R'}\NormalTok{) }\CommentTok{# this file has the rebound ANOVA}
\KeywordTok{source}\NormalTok{(}\StringTok{'Figures.R'}\NormalTok{) }\CommentTok{# this file creates the 4 figures}
\end{Highlighting}
\end{Shaded}

\section{Overview}\label{overview}

This document discusses the figures and statistics used to investigate
abrupt versus gradual motor learning. The main sections here will be
looking at the Order Effects, the Extent of Learning, and the Rebound.

\section{Plotting All of the Behavioural
Data}\label{plotting-all-of-the-behavioural-data}

\begin{Shaded}
\begin{Highlighting}[]
\KeywordTok{plotAllData}\NormalTok{()}
\end{Highlighting}
\end{Shaded}

\includegraphics{twoRate_notebook_files/figure-latex/unnamed-chunk-2-1.pdf}

\section{Order Effects of Within-Subjects
Design}\label{order-effects-of-within-subjects-design}

This was done only on data from the abrupt condition, with groups (30°
digitizing tablet, 60° digitalizing tablet, 30° VR setups) with the
blocks (first, second, last training) as our within-subject factors, and
order as our between-subject factors. The orange bars here represent the
mean reach deviations of the participants who performed the abrupt
condition first, and the blue bars represent the mean reach deviations
of the participants who performed the gradual condition first. All of
the data has been normalized to the size of the rotation.

\begin{Shaded}
\begin{Highlighting}[]
\KeywordTok{plotOrderData}\NormalTok{()}
\end{Highlighting}
\end{Shaded}

\includegraphics{twoRate_notebook_files/figure-latex/unnamed-chunk-3-1.pdf}

Here is the ANOVA for each of the tablet30 group, which showed no effect
of order.

\begin{Shaded}
\begin{Highlighting}[]
\KeywordTok{orderANOVA}\NormalTok{(}\DataTypeTok{group =} \StringTok{'tablet30'}\NormalTok{)}
\end{Highlighting}
\end{Shaded}

\begin{verbatim}
## $ANOVA
##        Effect DFn DFd          F            p p<.05        ges
## 2       order   1  28   0.723500 4.022137e-01       0.01547996
## 3       block   2  56 107.157288 7.193161e-20     * 0.59972241
## 4 order:block   2  56   1.105181 3.382646e-01       0.01521742
## 
## $`Mauchly's Test for Sphericity`
##        Effect         W         p p<.05
## 3       block 0.9488241 0.4920474      
## 4 order:block 0.9488241 0.4920474      
## 
## $`Sphericity Corrections`
##        Effect       GGe        p[GG] p[GG]<.05      HFe        p[HF]
## 3       block 0.9513156 5.137529e-19         * 1.018806 7.193161e-20
## 4 order:block 0.9513156 3.362093e-01           1.018806 3.382646e-01
##   p[HF]<.05
## 3         *
## 4
\end{verbatim}

Here is the ANOVA for each of the tablet60 group, which showed no effect
of order.

\begin{Shaded}
\begin{Highlighting}[]
\KeywordTok{orderANOVA}\NormalTok{(}\DataTypeTok{group =} \StringTok{'tablet60'}\NormalTok{)}
\end{Highlighting}
\end{Shaded}

\begin{verbatim}
## $ANOVA
##        Effect DFn DFd            F            p p<.05          ges
## 2       order   1  26  0.009481915 9.231753e-01       0.0002318619
## 3       block   2  52 88.074921744 2.006145e-17     * 0.5522311586
## 4 order:block   2  52  0.528464898 5.926413e-01       0.0073456288
## 
## $`Mauchly's Test for Sphericity`
##        Effect         W         p p<.05
## 3       block 0.8972168 0.2577612      
## 4 order:block 0.8972168 0.2577612      
## 
## $`Sphericity Corrections`
##        Effect       GGe        p[GG] p[GG]<.05       HFe        p[HF]
## 3       block 0.9067965 5.167944e-16         * 0.9709382 5.522909e-17
## 4 order:block 0.9067965 5.755789e-01           0.9709382 5.874898e-01
##   p[HF]<.05
## 3         *
## 4
\end{verbatim}

Here is the ANOVA for each of the VR30 group, which showed no effect of
order.

\begin{Shaded}
\begin{Highlighting}[]
\KeywordTok{orderANOVA}\NormalTok{(}\DataTypeTok{group =} \StringTok{'VR30'}\NormalTok{)}
\end{Highlighting}
\end{Shaded}

\begin{verbatim}
## Warning: Data is unbalanced (unequal N per group). Make sure you specified
## a well-considered value for the type argument to ezANOVA().
\end{verbatim}

\begin{verbatim}
## $ANOVA
##        Effect DFn DFd           F            p p<.05          ges
## 2       order   1  17 0.001798478 0.9666671996       6.138754e-05
## 3       block   2  34 9.774540542 0.0004429525     * 1.944044e-01
## 4 order:block   2  34 0.028268511 0.9721501280       6.974172e-04
## 
## $`Mauchly's Test for Sphericity`
##        Effect         W         p p<.05
## 3       block 0.9426206 0.6232977      
## 4 order:block 0.9426206 0.6232977      
## 
## $`Sphericity Corrections`
##        Effect       GGe        p[GG] p[GG]<.05      HFe        p[HF]
## 3       block 0.9457344 0.0005875801         * 1.060541 0.0004429525
## 4 order:block 0.9457344 0.9672409985           1.060541 0.9721501280
##   p[HF]<.05
## 3         *
## 4
\end{verbatim}

\section{Extent of Learning}\label{extent-of-learning}

Here we are checking if there are any differences in the extent of
learning in the abrupt and gradual conditions during the last training
and last reversal blocks. The data is shown for the 3 groups (30°
digitizing tablet, 60° digitalizing tablet, 30° VR setups). The orange
bars here represent the mean reach deviations of the abrupt condition,
and the blue bars represent the mean reach deviations of gradual
condition. All of the data has been normalized to the size of the
rotation.

\begin{Shaded}
\begin{Highlighting}[]
\KeywordTok{plotExtentofLearning}\NormalTok{()}
\end{Highlighting}
\end{Shaded}

\includegraphics{twoRate_notebook_files/figure-latex/unnamed-chunk-7-1.pdf}

Here is the ANOVA for each of the tablet30 group, which showed that
there were no significant differences in the abrupt and gradual
conditions.

\begin{Shaded}
\begin{Highlighting}[]
\KeywordTok{ExtentOfLearningANOVA}\NormalTok{(}\DataTypeTok{group =} \StringTok{'tablet30'}\NormalTok{)}
\end{Highlighting}
\end{Shaded}

\begin{verbatim}
## $ANOVA
##            Effect DFn DFd            F            p p<.05          ges
## 2           block   1  29 1.064469e+02 3.233667e-11     * 5.502381e-01
## 3       condition   1  29 1.389807e+00 2.480201e-01       6.367231e-03
## 4 block:condition   1  29 4.658796e-03 9.460508e-01       2.030507e-05
\end{verbatim}

Here is the ANOVA for each of the tablet60 group, which showed that
there were no significant differences in the abrupt and gradual
conditions.

\begin{Shaded}
\begin{Highlighting}[]
\KeywordTok{ExtentOfLearningANOVA}\NormalTok{(}\DataTypeTok{group =} \StringTok{'tablet60'}\NormalTok{)}
\end{Highlighting}
\end{Shaded}

\begin{verbatim}
## $ANOVA
##            Effect DFn DFd            F            p p<.05          ges
## 2           block   1  27 244.53594288 4.678244e-15     * 7.982554e-01
## 3       condition   1  27   0.01877024 8.920437e-01       9.148747e-05
## 4 block:condition   1  27   0.05619955 8.143963e-01       2.262949e-04
\end{verbatim}

Here is the ANOVA for each of the VR30 group, which showed that there
were no significant differences in the abrupt and gradual conditions.

\begin{Shaded}
\begin{Highlighting}[]
\KeywordTok{ExtentOfLearningANOVA}\NormalTok{(}\DataTypeTok{group =} \StringTok{'VR30'}\NormalTok{)}
\end{Highlighting}
\end{Shaded}

\begin{verbatim}
## $ANOVA
##            Effect DFn DFd           F            p p<.05          ges
## 2           block   1  18 76.64090547 6.635203e-08     * 0.5078841945
## 3       condition   1  18  0.11507302 7.383665e-01       0.0022687329
## 4 block:condition   1  18  0.05084943 8.241322e-01       0.0004706297
\end{verbatim}

\section{Rebound}\label{rebound}

Here we are checking if there are any differences in the rebound between
the abrupt and gradual conditions for the 3 groups (30° digitizing
tablet, 60° digitalizing tablet, 30° VR setups). The orange bars here
represent the mean reach deviations of the abrupt condition, and the
blue bars represent the mean reach deviations of gradual condition. All
of the data has been normalized to the size of the rotation.

\begin{Shaded}
\begin{Highlighting}[]
\KeywordTok{plotReboundData}\NormalTok{()}
\end{Highlighting}
\end{Shaded}

\includegraphics{twoRate_notebook_files/figure-latex/unnamed-chunk-11-1.pdf}

Here is the ANOVA for each of the tablet30 group, which showed that
there were no significant differences in the abrupt and gradual
conditions.

\begin{Shaded}
\begin{Highlighting}[]
\KeywordTok{getReboundANOVA}\NormalTok{(}\DataTypeTok{group =} \StringTok{'tablet30'}\NormalTok{)}
\end{Highlighting}
\end{Shaded}

\begin{verbatim}
## $ANOVA
##      Effect DFn DFd         F         p p<.05         ges
## 2 condition   1  29 0.1522538 0.6992422       0.001854785
\end{verbatim}

Here is the ANOVA for each of the tablet30 group, which showed that
there were no significant differences in the abrupt and gradual
conditions.

\begin{Shaded}
\begin{Highlighting}[]
\KeywordTok{getReboundANOVA}\NormalTok{(}\DataTypeTok{group =} \StringTok{'tablet60'}\NormalTok{)}
\end{Highlighting}
\end{Shaded}

\begin{verbatim}
## $ANOVA
##      Effect DFn DFd       F         p p<.05       ges
## 2 condition   1  28 1.37354 0.2510805       0.0220682
\end{verbatim}

Here is the ANOVA for each of the tablet30 group, which showed that
there were no significant differences in the abrupt and gradual
conditions.

\begin{Shaded}
\begin{Highlighting}[]
\KeywordTok{getReboundANOVA}\NormalTok{(}\DataTypeTok{group =} \StringTok{'VR30'}\NormalTok{)}
\end{Highlighting}
\end{Shaded}

\begin{verbatim}
## $ANOVA
##      Effect DFn DFd          F         p p<.05         ges
## 2 condition   1  18 0.01202616 0.9138893       0.000325402
\end{verbatim}


\end{document}
